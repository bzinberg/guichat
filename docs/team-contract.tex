\documentclass[11pt]{article}

\usepackage{amsmath,amsfonts,amssymb,amsthm}
\usepackage{fullpage}

\title{6.005 Project 2 Team Contract}
\author{Akashnil Dutta, Ben Zinberg, Vinay Mayar}

\newcommand{\question}[2]{{\noindent\bf #1}\newline #2 \newline}

\begin{document}
\maketitle

{\em Credit:} This contract is based on the team contract for Project 1 used by Ben Zinberg, Jacob Hurwitz and Steven Valdez.  Because the projects are similar in purpose, this contract resembles the former one very closely.

\section{Goals}
\question{What are the goals of the team?}{
To design and implement a working instant messenger.}

\question{What are your personal goals for this assignment?}{
To learn good coding practices and get a sense for the various trade-offs one must face in designing software (e.g., more features vs. easier-to-manage code base).}

\question{What kind of obstacles might you encounter in reaching your goals?}{
Disagreement within the team, a very difficult project, not enough time to work.  Ben is leaving on May 15 so we will need to be essentially done by the 14th.  On the bright side, this means we will be able to enter the class contest.}

\question{What happens if all of you decide you want to get an A grade, but because of time constraints, one person decides that a B will be acceptable?}{
The point of a contract is that we can't renege on our decisions later. If this situation arises, we would ask our TA for advice, since that appears to be the person in charge of reviewing the git history to make sure each team member contributes.}

\question{Is it acceptable for one or two team members to do more work than the others in order to get the team an A?}{
No.  You may not do someone else's assigned portion of the work unless the internal deadline for that part of the project has passed.  This is very important---doing someone else's work can potentially threaten their grade, not just yours.  If a team member misses his internal deadline, however, steps must be taken to ensure that the external deadline is met.  This means that if a task is not done by the internal deadline, the other two team members may take over on that task.  Hopefully, such measures will never have to be taken. (Note: we use the term internal deadlines to mean deadlines set by the team, and external deadlines to mean deadlines set by the course staff.)}

\section{Meeting Norms}
\question{Do you have a preference for when meetings will be held? Do you have a preference for where they should be held?}{
We all live within ten feet of each other, so it will be easy to communicate as much as we need.  ``Meetings'' will probably be held in the lounge in the evenings around dinner time.}

\question{How will you use the in-class time?}{
We've decided that our in-class time will be best spent focused on individually reviewing and working on the assignment.}

\question{How often do you think the team will need to meet outside of class? How long do you anticipate meetings will be?}{
Because we live so close together, there is no overhead involved in communication.  Thus, we will probably operate mostly by micro-meetings, with only two or three true meetings of maybe an hour each.  This is of course just an initial estimate; we'll see whether we end up needing more meeting time as the project goes on.}

\question{How will you record and distribute the minutes and action lists produced by each meeting?}{
If the proceedings of a meeting are sufficiently important or complex that we need minutes, we'll put them in the \texttt{docs} folder of our repo.  We will at least put some form of task list (including an indication of which group member is to do each thing) in the repo.}

\section{Work Norms}
\question{How much time per week do you anticipate it will take to make the project successful?}{
This is a 12-unit class, so probably 12 hours per week per person.}

\question{How will work be distributed?}{
Early on, we will split the work into three roughly equally sized parts.  This may be challenging to do, since the project does not naturally fit into a three-component blueprint; we will do our best to lay out the division of labor initially, but the plan may change somewhat as the project progresses. Each person will responsible for implementing his part, writing all associated tests, and making sure that this part interfaces correctly with the other two parts.}

\question{How will deadlines be set?}{
Collaboratively as a group, with results recorded in the \texttt{docs} folder.}

\question{How will you decide who should do which tasks?}{
Collaboratively as a group, with results recorded in the \texttt{docs} folder.}

\question{Where will you record who is responsible for which tasks?}{
In the \texttt{docs} folder.}

\question{What will happen if someone does not follow through on a commitment (e.g., missing a deadline, not showing up to meetings)?}{
We will attempt to contact that person to figure out what happened.}

\question{How will the work be reviewed?}{
We will review code after it's pushed, not before, since that's a bit easier. There is a git hook to email us when code is pushed; team members can reply to these emails with any concerns about the pushed code.}

\question{What happens if people have different opinions on the quality of the work?}{
That group member should explain to the other two members that there is a deficiency.  Once he has done this, the group will improve that deficiency, since we all care about the quality of our work.}

\question{What will you do if one or more team members are not doing their share of the work?}{
Talk to them, encourage them to do more, and if that doesn't work, talk to our TA.}

\question{How will you deal with different work habits of individual team members (e.g., some people like to get assignments done as early as possible; others like to work under the pressure of a deadline)?}{
Since we need to finish by May 14, hopefully we can stay ahead of the game.  Our intention is to have pacing taken care of by our internal deadlines -- as long as everything is done by our internal deadlines, individual work styles should not matter.}

\section{Decision Making}
\question{Do you need consensus (100\% approval of all team members) before making a decision?}{
We hope that no decisions are that contentious! In the rare event that a decision is, 2/3 is enough.}

\question{What will you do if one of you fixates on a particular idea?}{
If the idea has to do with implementation of his part of the project only, then that's his prerogative.  Anything that affects the other team members' work, though, needs to be a group decision.}

\end{document}
